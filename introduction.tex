\epigraph{Humanum est errare}{{\itshape Seneca the Elder}}
The European Commission states in\autocite{EU2019market} that: ``Cybersecurity
is one of the priority areas [\ldots] of the Commission initiative on ICT
Standards, which is part of the Digitising European
Industry\autocite{EU2019standard} strategy launched on 19 April 2016.  The aim
is to identify the essential ICT standards and present measures to accelerate
their development in support of digital innovations across the economy''. The
same document (i.e.\autocite{EU2019market}) states that ``The EU will invest up
to \euro450 million [\ldots], under its research and innovation programme
Horizon 2020''. The EU, in 2016 published a press release\autocite{EU2016press}
in which they present a strategy to invest \euro1.8 \emph{billion} to
``increase measures to address cyber threats''. The EU is not the only investor
in cybersecurity, most of the developed countries and several companies are
investing enormous amount of money towards various aspects of cybersecurity
(e.g. The US vulnerability databases\autocite{NIST2020NVD} maintained by the
National Institute of Standards and Technologies, i.e. NIST, of the US
Department of Commerce).

The cybersecurity industry is growing fast, e.g. as reported
in\autocite{Nasdaq2018market}. For example, in\autocite{Forbes2017market},
published by the Forbes, is stated that \euro5.3 billion of funding where
poured by venture capitalist into cybersecurity companies in 2018. The Forbes,
in the same article, also highlights another peculiar (as seemly contradictory)
trend: ``[\ldots] during the same time period, the number of cybersecurity
breaches increased exponentially''. The data reported by the NIST 
through the official CPE (Common Platform Enumeration) Dictionary Statistics 
on the NVD websites in \autocite{NIST2020CPEstatistics}, show that in 2016 the
number of reported vulnerabilities reported where around $~6000$ while in 2019
the number of vulnerabilities was above $16000$.  The scientific community also
reports similar findings.  In fact, in\autocite{Herley2009so}, Cormac Herley
(Microsoft Research) shows how basic cybersecurity principles (such as the
confidentiality benefit over the clear text for passwords typed into forms,
e.g. for logins in websites) are not fully understood or shared between the
cybersecurity research community\autocite{Nielsen2009stop}.  The lack of
understanding of basic security principle, the inverse proportionality between
investments in cybersecurity and the number of reported vulnerabilities year
after year, can be linked to the lack of a foundational theory on
cybersecurity, as already highlighted by Cormac Herley
in\autocite{Herley2016unfalsifiability}. 

Another way of looking at the same problem is by analyzing the different
definitions of security with respect to the scientific method of enquiry
applied to get to the definition itself. In the following we categorize the
related work into three categories, as defined by Sextus Empiricus in
\autocite{Empiricus1990Pyrrhonism}.  To the best of our knowledge, there is no
evidence that this categorization isn't complete.  ``The natural result of any
investigation is that the investigators either discover the object of search or
deny that it is discoverable and confess it to be in-apprehensible or persist
in their search. [\ldots] This is probably
why''\autocite{Empiricus1990Pyrrhonism}: 
\begin{itemize}
	\item The \emph{dogmatists} ``have claimed to have discovered the truth'' on what cybersecurity is
		\begin{itemize}
			\item Wikipedia defines cybersecurity in
				\autocite{wiki-cybersecurity} as the protection
				of computer systems and networks from the theft
				of or damage to their hardware, software, or
				electronic data, as well as from the disruption
				or misdirection of the services they provide
		\end{itemize}
	\item The \emph{academics} ``have asserted that it cannot be apprehended''
		\begin{itemize}
			\item Eugene H. Spafford, Professor at Purdue
				University, defines cybersecurity as follow.
				``The only truly secure system is one that is
				powered off, cast in a block of concrete and
				sealed in a lead-lined room with armed guards —
				and even then I have my doubts.''
				\autocite{Spafford2019Quotes}
		\end{itemize}
	\item The \emph{skeptics} ``go on inquiring''
		\begin{itemize}
			\item Cormac Herley reaches the conclusion that
				cybersecurity has no definition. ``There is an
				inherent asymmetry in computer security: things
				can be declared insecure by observation, but
				not the reverse. There is no test that allows
				us to declare an arbitrary system or technique
				secure. This implies that claims of necessary
				conditions for security are unfalsifiable. ''
				\autocite{Herley2016unfalsifiability}
		\end{itemize}
\end{itemize}

In this article, we give the first scientific theory (to the best of our
knowledge) on security.

\Paragraph{Structure} In Section~\ref{sec:problem} we define and formalize the
problem statement.  In Section~\ref{sec:theory} we outline our security theory,
and in Section~\ref{sec:prototest} we describe the implementation of the theory
and some empirical tests of the theory.  Finally, in Section~\ref{sec:related}
we conclude the paper with an overview of the related work.
