\epigraph{Humanum est errare}{{\itshape Seneca the Elder}}
The cybersecurity industry is growing fast (see Appendix~\ref{app:economy} for
a longer economic motivation), e.g. as reported in\autocite{Nasdaq2018market}.
For example, in\autocite{Forbes2017market}, published by the Forbes, is stated
that \euro5.3 billion of funding where poured by venture capitalist into
cybersecurity companies in 2018. The Forbes, in the same article, also
highlights another peculiar (as seemly contradictory) trend: ``[\ldots] during
the same time period, the number of cybersecurity breaches increased
exponentially''. The data reported by the NIST through the official CPE (Common
Platform Enumeration) Dictionary Statistics on the NVD websites in
\autocite{NIST2020CPEstatistics}, show that in 2016 the number of reported
vulnerabilities reported where around $~6000$ while in 2019 the number of
vulnerabilities was above $16000$.  The scientific community also reports
seemingly unjustifiable findings.  In fact, in\autocite{Herley2009so}, Cormac
Herley (Microsoft Research) shows how basic cybersecurity principles (such as
the confidentiality benefit over the clear text for passwords typed into forms,
e.g. for logins in websites) are not fully understood or shared between the
cybersecurity research community\autocite{Nielsen2009stop}.  The lack of
understanding of basic security principle, the inverse proportionality between
investments in cybersecurity and the number of reported vulnerabilities year
after year, can be linked to the lack of a foundational theory on
cybersecurity, as already highlighted by Cormac Herley
in\autocite{Herley2016unfalsifiability}.  An important issue raised by Herley
is that the methodology applied to by the security research community is quite
often implicit and not always scientifically ``correct''. Herley equates
correct to what Popper defined in \autocite{popper1962conjectures} even if, we
believe, the comments made by Hintikka in \autocite{Hintikka1993Information}
should be taken into account. Therefore, without discussing what correct and
non-correct is, we start by explicitly mention our method of enquiry. 

\Paragraph{Methodology} The method applied in our line of research follows what
(to the best of our knowledge) the scientific method mandates. We start, in
Section~\ref{sec:problem}, by detailing the problem statement, reporting a
literature review on the main concepts and definitions related to security.  We
formulate (inductively)\footnote{``So, whenever they argue ``Every man is an
animal and Socrates is a man; therefore Socrates is an animal,'' proposing to
deduce from the universal proposition ``every man is an animal'' the particular
proposition ``Socrates therefore is an animal,'' which in fact goes (as we have
mentioned) to enstablish by way of induction the universal proposition, the
fall into the error of circular reasoning, since they are enstablishing the
universal proposition inductively by means of each of the particulars and
deducing the particular proposition from the universal syllogistically.''
Sextus Empiricus, Outlines of Pyrrhonism II-195
\autocite{Empiricus1990Pyrrhonism}} a security hypothesis, based on our review,
in Section~\ref{sec:glossary}; which we use to propose a theory on system security in
Section~\ref{sec:theory}. We apply our theory to an abstract and standard (i.e.
based on standards) secure process development lifecycle, following the
principle of requirement engineering in Section~\ref{sec:engineering} (with an
ad-hoc running example, based on the SWaT testbed \autocite{Mathur2016swat}).
We implement, in Section~\ref{sec:tool}, the theory in a prototype tool that can be used for the risk
assessment of a design of cyber-physical systems.
