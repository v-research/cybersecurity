\epigraph{Humanum est errare}{{\itshape Seneca the Elder}}
The cybersecurity industry is growing fast (see Appendix~\ref{app:economy} for
a longer economic motivation), e.g. as reported in\autocite{Nasdaq2018market}.
For example, in\autocite{Forbes2017market}, published by the Forbes, is stated
that \euro5.3 billion of funding where poured by venture capitalist into
cybersecurity companies in 2018. The Forbes, in the same article, also
highlights another peculiar (as seemly contradictory) trend: ``[\ldots] during
the same time period, the number of cybersecurity breaches increased
exponentially''. The data reported by the NIST through the official CPE (Common
Platform Enumeration) Dictionary Statistics on the NVD websites in
\autocite{NIST2020CPEstatistics}, show that in 2016 the number of reported
vulnerabilities were around $~6000$ while in 2019 the number of
vulnerabilities was above $16000$.  The scientific community also reports
seemingly unjustifiable findings.  In fact, in\autocite{Herley2009so}, Cormac
Herley (Microsoft Research) shows how basic cybersecurity principles (such as
the confidentiality benefit over the clear text for passwords typed into forms,
e.g. for logins in websites) are not fully understood or shared between the
cybersecurity research community\autocite{Nielsen2009stop}.  The lack of
understanding of basic security principle, the inverse proportionality between
investments in cybersecurity and the number of reported vulnerabilities year
after year, can be linked to the lack of a foundational theory on
cybersecurity, as already highlighted by Cormac Herley
in\autocite{Herley2016unfalsifiability}.  An important issue raised by Herley
is that the methodology applied by the security research community is quite
often implicit and not always scientifically ``correct''. Herley equates
correct to what Popper defined in \autocite{popper1962conjectures} even if, we
believe, the comments made by Hintikka in \autocite{Hintikka1993Information}
should be taken into account. Therefore, we start by explicitly mention our method of enquiry. 

\Paragraph{Methodology} The method applied in our line of research follows what
(to the best of our knowledge) the scientific method mandates. We start, in
Section~\ref{sec:problem}, by detailing the problem statement, reporting a
literature review on the main concepts and definitions related to security.  We
formulate 
a security hypothesis %, based on our review,
in Section~\ref{sec:glossary}; which we use to propose a theory on system
security in Section~\ref{sec:theory}. In Section~\ref{sec:engineering}, we
apply our theory to an abstract and standard (i.e.  based on standards) secure
process development lifecycle, following the principle of requirement
engineering  (with an ad-hoc running example, based on the SWaT testbed
\autocite{Mathur2016swat}). This application shows how our theory can be used
to predict all of the possible security weaknesses of a system, allowing the
falsification of our theory.  In fact, if any security weaknesses were to be
found in a system and not predicted by our theory, the theory could be declared
incomplete.  Similarly, if a security weakness would be predicted by our theory
but found to be impossible to have our theory could be declared as wrong.  In
order to apply our theory we implemented, as described in
Section~\ref{sec:tool}, the theory in a prototype tool that can be used for the
risk assessment of a design of cyber-physical systems.
