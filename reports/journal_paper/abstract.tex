The objective of this research is the development of a theory that defines (all
and only) the possible insecurity and security configurations of any abstract
system. The theory is structured upon other theories that defines how a
component of a system can be abstracted into an agent, defining how agents can
be formalized (both syntactically and semantically) to describe an abstract
system, such as a graph. Some of these theories (e.g. used for the semantic
definition of the abstract system) are the epistemological definition of
knowledge, the Belief-Desire-Intent and the Assertion-Belief-Fact framework of
reference, mereology, and topological structure. We argue that a mereology is
the most appropriate abstract underlying structure, due to its generality, for
defining the expressiveness of the system abstraction.  Furthermore, a
mereology allows us to define an ontology rather than a taxonomy.  We also
correlate different abstractions of the system to the TRL and the engineering
V-model. 

We implemented a formal theory (of axioms) of a mereotopology, and of the
Region Connection Calculus (RCC3 and RCC5) in a Python program that uses the Z3
SMT solver. The results show that a single component (i.e.  agent) of an
abstract system has a definite number of  different insecurity configurations
(e.g. 53 using RCC5 over a topological structure) and only 1 secure (i.e.
expected) configurations. The configurations are reported as models satisfying
the abstract system semantics. The implementation allows us to  apply our
theory to system engineering and showing concrete applications of our theory to
the risk assessment of an ad-hoc system.
