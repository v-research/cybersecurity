The European Commission states in\autocite{EU2019market} that: ``Cybersecurity
is one of the priority areas [\ldots] of the Commission initiative on ICT
Standards, which is part of the Digitising European
Industry\autocite{EU2019standard} strategy launched on 19 April 2016.  The aim
is to identify the essential ICT standards and present measures to accelerate
their development in support of digital innovations across the economy''. The
same document (i.e.\autocite{EU2019market}) states that ``The EU will invest up
to \euro450 million [\ldots], under its research and innovation programme
Horizon 2020''. The EU, in 2016 published a press release\autocite{EU2016press}
in which they present a strategy to invest \euro1.8 \emph{billion} to
``increase measures to address cyber threats''. The EU is not the only investor
in cybersecurity, most of the developed countries and several companies are
investing enormous amount of money towards various aspects of cybersecurity
(e.g. The US vulnerability databases\autocite{NIST2020NVD} maintained by the
National Institute of Standards and Technologies, i.e. NIST, of the US
Department of Commerce).
