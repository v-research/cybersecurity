\documentclass{article}
\usepackage{mathtools}
\usepackage{caption}
\usepackage{subcaption}
\usepackage{enumitem}
\usepackage{multicol}
\usepackage[toc,page]{appendix}
\usepackage[gen]{eurosym}
\usepackage{tabularx}
\usepackage{epigraph}
\usepackage{textcomp}
\usepackage[english]{babel}
\usepackage[utf8]{inputenc}
\usepackage{amsmath}
\usepackage{amsthm}
\usepackage{amssymb}
\usepackage{hyperref}
\usepackage{csquotes}% Recommended
\usepackage{rotating}
\usepackage[style=numeric]{biblatex}
% \bibliography{<mybibfile>}% ONLY selects .bib file; syntax for version <= 1.1b
\addbibresource{literature.bib}% Syntax for version >= 1.2

\usepackage{graphicx}
\graphicspath{{./images/}}

\usepackage{fancyhdr}
\pagestyle{fancy}
\lhead{}
\chead{}
\rhead{}
\lfoot{}
\cfoot{\thepage}
\rfoot{Property of V-Research S.r.l.}
\renewcommand\headrulewidth{0pt}
\renewcommand\footrulewidth{0pt}

\usepackage[colorinlistoftodos]{todonotes}

\newcommand{\Paragraph}[1]{\smallskip\noindent{\bf #1.}}
\newcommand{\fixnote}[2]{\textbf{\color{red}{FIX}}\footnote{{\bf #1:} #2}}
\newcommand{\fix}[2]{{\color{red} {\bf tofix:} #2}}
% \renewcommand{\fixnote}[2]{}
% \renewcommand{\fix}[2]{}

\theoremstyle{definition}
\newtheorem{definition}{Definition}[section]
\theoremstyle{corollary}
\newtheorem{corollary}{Corollary}[section]
\theoremstyle{lemma}
\newtheorem{lemma}{Lemma}[section]
\theoremstyle{theorem}
\newtheorem{theorem}{Theorem}[section]
\theoremstyle{theorem}
\newtheorem{example}{Example}

\usepackage{authblk}
\date{}                     %% if you don't need date to appear
\setcounter{Maxaffil}{0}
\renewcommand\Affilfont{\itshape\small}

% for fancy table with rotated column titles
\usepackage{adjustbox}
\newcolumntype{R}[2]{%
    >{\adjustbox{angle=#1,lap=\width-(#2)}\bgroup}%
    l%
    <{\egroup}%
}
\newcommand*\rot{\multicolumn{1}{R{45}{0.01em}}}% no optional argument here, please!
\newcommand*\rota{\multicolumn{1}{R{90}{0.01em}}}% no optional argument here, please!
\def\hb{\hbox to 10.7 cm{}}

% circles
\usepackage{wasysym}
\newcommand{\Tdot}{$\CIRCLE$}
\newcommand{\Thdot}{$\LEFTcircle$}
\newcommand{\Twdot}{$\Circle$}

% formulas
\newcommand{\kframe}{\mathbf{K}}
\newcommand{\bframe}{\mathbf{B_K}}
\newcommand{\possibleworlds}{G}
\newcommand{\modalrelation}{R}
\newcommand{\actualworld}{\omega*}
\newcommand{\world}{\omega}
\newcommand{\World}{\Omega}
\newcommand{\kmodel}{M}
\newcommand{\interpretation}{\sigma}
\newcommand{\knowledge}[1]{\mathbb{K}_{#1}}
\newcommand{\knows}[2]{K_{#1}#2}
\newcommand{\belief}[1]{\mathbb{B}_{#1}}
\newcommand{\believe}[2]{B_{#1}#2}
\newcommand{\region}{\chi}
\newcommand{\Region}{\mathrm{X}}
\newcommand{\agentuniverse}{\mathcal{A}}
\newcommand{\agent}{a}
\newcommand{\follang}{\mathcal{L}}
\newcommand{\weakness}{w}
\newcommand{\Weakness}{W}
\newcommand{\vulnerability}{v}
\newcommand{\Vulnerability}{V}
%rcc5
\newcommand{\rcc}{rcc}
\newcommand{\connects}[2]{C(#1,#2)}
\newcommand{\disconnected}[2]{\neg C(#1,#2)}
\newcommand{\partof}[2]{P(#1,#2)}
\newcommand{\overlaps}[2]{O(#1,#2)}
\newcommand{\eq}[2]{EQ(#1,#2)}
\newcommand{\pp}[2]{PP(#1,#2)}
\newcommand{\po}[2]{PO(#1,#2)}
\newcommand{\ppi}[2]{PPi(#1,#2)}
\newcommand{\dr}[2]{DR(#1,#2)}


\makeindex 

\begin{document}

\title{On Cybersecurity\\Science and Engineering\footnote{Confidential, restricted to the authors. Property of V-Research S.r.l.}}
\author[1]{Francesco Beltramini}
\author[1]{Marco Rocchetto}
\affil[1]{V-Research, Verona, Italy}

\maketitle

\begin{abstract}
 The objective of this research is to develop of a theory that defines
(all and only) the possible insecurity and security configurations of
any abstract system. The theory is structured upon other theories that 
defines how a component of a system can be abstracted into an agent, defining how
agents can be formalized (both syntactically and semantically) to
describe an abstract system, such as a graph. Some of these theories
(e.g. used for the semantic
definition of the abstract system) are the epistemological definition
of knowledge, the Belief-Desire-Intent and the Assertion-Belief-Fact 
framework of reference, 
mereology, and topological structure. We argue that a mereology is the
most appropriate abstract underlying structure, do to its generality,
for defining the expressiveness of the system abstraction.  Furthermore, a
mereology allows us to define an ontology rather than a taxonomy.  We
also correlate different abstractions of the system to the TRL and the
engineering V-model. 

We implemented a formal theory (of axioms) of a mereotopology, and of
the Region Connection Calculus (RCC3 and RCC5) in a Python program that
uses the Z3 SMT solver. The results show that a single component (i.e.
agent) of an abstract system has a definite number of  different
insecurity configurations (e.g. 53 using RCC5 over a topological
structure) and only 1 secure (i.e. expected) configurations. The
configurations are reported as models satisfying the abstract system
semantics. 

We considered the philosophical definition of truth behind our
approach, rejecting ``proof'' by induction from partial empirical evidences.
Our theory can be applied to system engineering and we show a concrete
application of our theory to the risk assessment of an ad-hoc system.
Finally, we provide a number of ideas to support the engineering of
secure systems (e.g. purely cyber or cyber-physical).

\end{abstract}
\newpage

\section{Introduction}\label{sec:intro}
\epigraph{Humanum est errare}{{\itshape Seneca the Elder}}
The cybersecurity industry is growing fast (see Appendix~\ref{app:economy} for
a longer economic motivation), e.g. as reported in\autocite{Nasdaq2018market}.
For example, in\autocite{Forbes2017market}, published by the Forbes, is stated
that \euro5.3 billion of funding where poured by venture capitalist into
cybersecurity companies in 2018. The Forbes, in the same article, also
highlights another peculiar (as seemly contradictory) trend: ``[\ldots] during
the same time period, the number of cybersecurity breaches increased
exponentially''. The data reported by the NIST through the official CPE (Common
Platform Enumeration) Dictionary Statistics on the NVD websites in
\autocite{NIST2020CPEstatistics}, show that in 2016 the number of reported
vulnerabilities were around $~6000$ while in 2019 the number of
vulnerabilities was above $16000$.  The scientific community also reports
seemingly unjustifiable findings.  In fact, in\autocite{Herley2009so}, Cormac
Herley (Microsoft Research) shows how basic cybersecurity principles (such as
the confidentiality benefit over the clear text for passwords typed into forms,
e.g. for logins in websites) are not fully understood or shared between the
cybersecurity research community\autocite{Nielsen2009stop}.  The lack of
understanding of basic security principle, the inverse proportionality between
investments in cybersecurity and the number of reported vulnerabilities year
after year, can be linked to the lack of a foundational theory on
cybersecurity, as already highlighted by Cormac Herley
in\autocite{Herley2016unfalsifiability}.  An important issue raised by Herley
is that the methodology applied by the security research community is quite
often implicit and not always scientifically ``correct''. Herley equates
correct to what Popper defined in \autocite{popper1962conjectures} even if, we
believe, the comments made by Hintikka in \autocite{Hintikka1993Information}
should be taken into account. Therefore, we start by explicitly mention our method of enquiry. 

\Paragraph{Methodology} The method applied in our line of research follows what
(to the best of our knowledge) the scientific method mandates. We start, in
Section~\ref{sec:problem}, by detailing the problem statement, reporting a
literature review on the main concepts and definitions related to security.  We
formulate 
a security hypothesis %, based on our review,
in Section~\ref{sec:glossary}; which we use to propose a theory on system
security in Section~\ref{sec:theory}. In Section~\ref{sec:engineering}, we
apply our theory to an abstract and standard (i.e.  based on standards) secure
process development lifecycle, following the principle of requirement
engineering  (with an ad-hoc running example, based on the SWaT testbed
\autocite{Mathur2016swat}). This application shows how our theory can be used
to predict all of the possible security weaknesses of a system, allowing the
falsification of our theory.  In fact, if any security weaknesses were to be
found in a system and not predicted by our theory, the theory could be declared
incomplete.  Similarly, if a security weakness would be predicted by our theory
but found to be impossible to have our theory could be declared as wrong.  In
order to apply our theory we implemented, as described in
Section~\ref{sec:tool}, the theory in a prototype tool that can be used for the
risk assessment of a design of cyber-physical systems.


\section{Problem Statement}\label{sec:problem}
\begin{figure}[t]
	\centering
	\includegraphics[width=.6\textwidth]{protocol-example.pdf}
	\caption{Abstraction of an ad-hoc esemplificative protocol execution}
	\label{fig:protocol-example}
\end{figure}

In\autocite{Herley2016unfalsifiability}, Cormac Herley explores what he calls
``an asymmetry in computer security'', which he defines as follows: ``Things
can be declared insecure by observation, but not the reverse. There is no
observation that allows us to declare an arbitrary system or technique
secure''. Herley then uses this argument to show that ``claims that any measure
is necessary for security are empirically unfalsifiable''. Given that, any
theory which is not falsifiable by an empirical experiment is well
known\footnote{``A theory which is not refutable by any conceivable event is
nonscientific. Irrefutability is not a virtue of a theory (as people often
think) but a vice.'' -- Karl Popper, Conjectures and
Refutations\autocite{popper1962conjectures}} to be nonscientific (i.e.
unfalsifiability is a fallacy of a theory), Herley concludes that there is no
scientific theory on cybersecurity; which means that cybersecurity lays in the
realm of pseudo-sciences\autocite{Herley2016usenixvideo}.  Herley, e.g.
in\autocite{Herley2017justifying}, discusses the implications of a
nonscientific approach to cybersecurity, and highlights the tremendous impact on
all the scientific research and engineering of systems; leading often to
terrorism and wars, and wasting of resources in useless protections or
overspending.  While the criticism is investigated
in\autocite{Herley2016unfalsifiability}, no solution is provided.  On the
contrary, the goal of this work \emph{is} to lay the foundations of a
scientific cybersecurity theory. Furthermore, in Section~\ref{sec:sicurezza},
we consider the problem raised by Herley not confined to ``computer security''
but to any abstract system (so that our theory may hold for any sound
implementation such as networks, mechanical, cyber, or cyber-physical system,
or even a single computer or a single device such as an hard-drive).  There is
also an apparent inconsistency in\autocite{Herley2016unfalsifiability} that we
seek to clarify before following (as we agree) the scientific path draw by
Herley: cybersecurity is defined as an abstract property in many formal
approaches to the investigation of the security of systems, and the security of the design of a
formally verified protocol is indeed falsifiable (against the security properties verified).  For example, in the protocol
verification community, security is often defined as a formalization of the
high-level properties confidentiality, integrity, and availability. The problem
in such approaches is not the definition of what cybersecurity is, but 
generality of the results since they takle a specific step (and not the first) of the engineering of system.
Therefore, the theories underlying the verification are based on 
assumptions which non-evidently apply to a general security theory (since, as mentioned before,
there is no sucha theory).
As an example, the so called Dolev-Yao attacker model\footnote{For the sake of
simplicity, the Dolev-Yao attacker can be considered as an abstraction of an
active attacker who controls the network but cannot break
cryptography.}\autocite{Dolev1983security})
that only applies to specific instances (often called scenarios) and
abstraction of the protocol. This, in turn, creates a false sense of security
since requires non-justifiable assumptions on the abstraction of the system of which security
is verified. As an example, for the formal security verification of the system
in Figure~\ref{fig:protocol-example}, a formalized scenario needs to be defined
by a modeler who chooses (among others): (i) a scope of the formalization (e.g.
excluding the server that distributes the public key is often done when
verifying the security of authentication protocols), (ii) the number of
sessions (even tough some approaches do reason on an infinite number of
sessions such as\autocite{Escobar2007maudenpa}), (iii) honesty/dishonesty of
the peers (e.g.  in the ASLan++ language\autocite{Oheimb2010aslan++}), and (iv)
the abstraction of the cryptographic primitives (e.g.  ProVerif vs
CryptoVerif\autocite{Blanchet2017symbolic}).  While many tools and theories
improves issues when discovered, there is no agreement on which should be
the definitive approach if any and, more importantly, some of the choices made by the modelers/engineers using those formal apporaches may
completely change the results of the formal verification of the system (an interesting example on how difficult it is to 
even just compare the approaches is given in \autocite{Cremers2009comparing}). For
example, under the perfect cryptography assumption\footnote{As defined
in\autocite{Rocchetto2016cpdy}: ``In the so called perfect cryptography
assumption, the security encryption scheme is suppose to be perfect, without
any exploitable flaw, and so the only way for the attacker to decrypt a message
is by using the proper key. That assumption is widely accepted in the security
protocol community, and most of the formal reasoning tools for the analysis of
security protocols abstract away the mathematical and implementation details of
the encryption
scheme\autocite{Turuani2006clatse,Basin2005ofmc,Armando2016satmc,Rocchetto2017interpolation}''}
and assuming that no violation to any security property is done after message
I); in Figure~\ref{fig:protocol-example}, the freedom of choosing the scope
determines that the flaws related to the dishonest impersonation of the Server
may or may not be considered in the verification process.  This choice has
tremendous impact on the focus and findings of the verification of the security
of the protocol.  While this may seem to turn upon minutiae and foreseeable,
this highlights the false sense of security that may derive from a
non-falsifiable theory of system security\footnote{``To the superficial
observer, the analysis of these forms seems to turn upon minutiae. It does in
fact deal with minutiae, but they are of the same order as those dealt with in
microscopic anatomy.'' -- Karl Marx, Capital Volume 1, 1867}.

\subsection{Sicurezza: Safety and Security}\label{sec:sicurezza}
In most of the natural languages, and in Italian too, the concepts of safety
and security are not syntactically differentiated and both terms (safety and
security) are expressed by the same word, e.g. sicurezza in Italian.  A
semantic distinction between safety and security is correlated to a
belief\footnote{A belief has to be intended as a proposition which is supposed
to be true by the majority of people in our society, without a scientific
underlying theory but based on partial empirical evidences or inductive reasoning based on partial empirical evidences.} that
safety deals with \emph{accidents} (i.e. an unfortunate incident) posed by the
natural environment (e.g. natural events such as wearing of hardware
components) while security deals with \emph{incidents} posed by mankind (e.g.
attackers and bugs).  The fundamental difference between nature and mankind (and,
in turn, between safety and cybersecurity) is believed to be on the different
intents\footnote{``The belief–desire–intention software model (BDI) is a
software model developed for programming intelligent
agents.''\autocite{wiki-bdi}. In the BDI model, the intents represents the
deliberative state of an agent which determines the choice of that agent on
what to do.} (accidents are unfortunate while incidents are not) of the causes
that generates a threat; namely, nature is believed not to have malicious
intents (but unfortunate causes-effects) while threats generated by mankind are
believed to be malicious\footnote{Of course, logical flaws or bugs may be
introduced by other means (e.g. ignorance) without explicit malicious intents,
but the exploitation of those flaws is considered (for now, and detailed
afterwards in the article) malicious, and then we consider any vulnerability to
be malicious even if due to the lack of skills.}.
An overview on the aforementioned aspects of safety and security is depicted in
Figure~\ref{fig:safety-security} and is used as a baseline for a definition of
the terms that structure our current understanding of safety and security. 
\begin{itemize}
	\item \emph{Mankind} ``refers collectively to humans''
\autocite{wiki-mankind}, while the concept of \emph{Nature} is
		related ``to the intrinsic characteristics that plants,
		animals, and other features of the world develop of their own
		accord'' (e.g. the physical universe)\autocite{wiki-nature}. 
		\begin{itemize}
			\item So far, we have used several terms to refer to an
				\emph{attacker}, i.e. threat agent or threat source,
				considering those terms to be semantically
				equivalent.  This ``shallowness'' has raised form the
				necessity of properly citing the different sources, but,
				in the reminder of this paper, we consider the
				Causality principle to be the \emph{threat
				source}, Nature or Mankind to be the
				\emph{threat agents} and an \emph{attacker} as
				a specific malicious threat agent which materializes a
				threat.
		\end{itemize}
	\item \emph{Vulnerability}\footnote{The term vulnerability is not
		present in the Encyclopedia of Cryptography and Security, while
		it is used in 12 entries (such as in the definition of
		``penetration testing''\autocite{caddy2005pentest})
		highlighting how commonly this word is used without a proper
		supporting semantics.}, as defined in\autocite{cnssi20104009}
		(and adopted in\autocite{nist2013800-53}), is ``weakness in an
		information system, system security procedures, internal
		controls, or implementation that could be exploited by a threat
		source''. On the one hand, the definition is broad to enclose
		as much causes (that generates a vulnerability) as possible; on
		the other hand, it derives from empirical evidences (which
		should be considered beliefs\footnote{``For this view, that
		\emph{That Which Is Not} exists, can never predominate. You
		must debar your thought from this way of search, nor let
		ordinary experience in its variety force you along this way,
		(namely, that of allowing) the eye, sightless as it is, and the
		ear, full of sound, and the tongue, to rule; but (you must)
		judge by means of the Reason (Logos) the much-contested proof
		which is expounded by me.'' -- Parmenides of Elea, On Nature
		(circa 500 B.C.), fragments B7.1–8.2
		\autocite{Hakim2016philosophy}} since they are partial results in nature) 
		while a vulnerability should
		be defined in a way that is empirically falsifiable. This means
		that the term vulnerability should have a complete and sound
		definition, so that no other causes (e.g.  other sources) but
		the ones in the definition are responsible for a vulnerability.
		Furthermore, the term ``threat sources'' used in the definition
		in\autocite{cnssi20104009} may be identified with both Nature
		and Mankind, not differentiating between safety and security.
		\fix{mr}{In Definition~\ref{def:vulnerability}}, we provide a formal
		theory of vulnerability (so that the scientific community can
		identify tests for the completeness and soundness of the
		definition itself).
\end{itemize}

\begin{figure}[t]
	\centering
	\includegraphics[width=\textwidth]{safety-security.pdf}
	\caption{Overview of keywords related to security and safety}
	\label{fig:safety-security}
\end{figure}

Most of the safety-preserving principles in the field of engineering of
safety-critical cyber-physical systems (such as elevators and aircraft), upon
which safety requirements are defined (e.g. in standards such as the IEC 61508 or 61511\autocite{IEC201761511}), 
are based on empirical tests and measurements (therefore should be considered hypothesis and not definitions). While reasoning by
induction based on the empirical observation should be avoided, since it may
easily lead to false beliefs, this approach is
often justified by the supposed impossibility of defining a theory that
correctly predicts failures which, in turn, pose hazards to a system. 
To the best of our knowledge, and supported by\autocite{Herley2016unfalsifiability}, the
correlation between predictability of environment and believed unpredictability
of attackers has not been correlated to a theory on cybersecurity. 
Therefore, inductive research efforts in
predicting malicious effects are accepted (and published) in scientific
conferences (e.g.~\autocite{Rocchetto2014CSRF}). A failure of a wire due to
environment (e.g. due to humidity, dust, heat \&c) is defined from empirical
evidences and processes have been standardized to test qualities of hardware components.
This process completely breaks down when a malicious environment (i.e. an attacker)
is considered instead of the (supposedly honest and predictable)
natural environment. Therefore, the same approach that is in use for safety,
seems not to be applicable for security (e.g. for security testing).

Going back to Figure~\ref{fig:safety-security}, a vulnerability does not
necessarily become a threat for the system, unless exploited ``through a
channel that allows the violation of the security policy
[\ldots]''\autocite{cnssi20104009}. For example, a software or procedure that takes
advantage of the vulnerability causing an \emph{attack} to the system may
result in several correlated incidents and threats.  The process of
exploitation of a defect as a vulnerability is reported in
Figure~\ref{fig:safety-security} such that the difference between exploit and failure,
and attack and accident is to be found just in the maliciousness of the intents
that causes this process (i.e. excluding the intent, the terms are just syntactic transformation from a vulnerability to defect, from
accident to incident). In the following, we conclude the informal definition of
the terms that we used in this section and in Figure~\ref{fig:safety-security}.

\begin{itemize}
	\item \emph{Causality} refers to the causality principle; defined
		in\autocite{Spirkin1983Dialectical} as ``Causality is a genetic
		connection of phenomena through which one thing (the cause)
		under certain conditions gives rise to, causes something else
		(the effect). The essence of causality is the generation and
		determination of one phenomenon by another. In this respect,
		causality differs from various other kinds of connection, for
		example, the simple temporal sequence of phenomena, of the
		regularities of accompanying processes''.
	\item An \emph{Exploit}\footnote{We note that the term exploit is only
		used as a verb in\autocite{ISO2009information}} ``[\ldots]
		(from the English verb to exploit, meaning to use something to
		one’s own advantage) is a piece of software, a chunk of data,
		or a sequence of commands that takes advantage of a bug or
		vulnerability to cause unintended or unanticipated behavior to
		occur on computer software, hardware, or something electronic
		(usually computerized).''\autocite{wiki-exploit}.
	\item An \emph{Attack}, as defined by the International Standard
		ISO/IEC 27000 is an ``attempt to destroy, expose, alter,
		disable, steal or gain unauthorized access to or make
		unauthorized use of an asset''; where an \emph{Asset} is
		``anything that has value to the organization''. We note that for
		the purpose of this article, we do not want to focus on a specific
		organization or business to define asset but, in general, on any 
		abstract organization (e.g. a company or a society).
		We do not consider ethical hackers as attacking a system. 
		In fact, we consider the term \emph{hack} as
		non-malicious (as, e.g. in \autocite{Stallman2002hacker}).
	\item A \emph{Threat}, as defined in\autocite{cnssi20104009}, is ``Any
		circumstance or event with the potential to adversely impact
		organizational operations (including mission, functions, image,
		or reputation), organizational assets, individuals, other
		organizations, or the Nation through an information system via
		unauthorized access, destruction, disclosure, modification of
		information, and/or denial of service''.
	\item \emph{Defect}, ``anything that renders the product not reasonably
		safe''\autocite{Robinson2019writing} (i.e. a characteristic of
		an object which hinders its proper usability).
	\item \emph{Failure}, as defined in\autocite{Merriam2020failure} as ``a state of
		inability to perform a normal function''. The term is
		structured and detailed in
		\autocite{cnssi20104009,iet2017glossary} but relying on an
		abstract notion of failure without a specific definition.
	\item \emph{Hazard}, ``a potential source of
		harm''\autocite{iet2017glossary}.
\end{itemize}
We now proceed to the formalization of the security concepts introduced in this
section. The main outcome of our formalization is that there is no evidence
that security must necessarily by linked to the malicious intent of an agent.
While this is the (only?) hypothesis behind security models and engineering
processes we start from a different hypothesis; system \emph{in-security} and
\emph{security} should not be searched on those events who maliciously produces
negative effects on a systems (incidents), but in the potential designs and
implementations due to vague specifications. Therefore, we will show that
security-by-design with a requirement-based engineering process is to be
preferred. 

The line of reasoning that we apply works as follows. The
epistemologic concept of Knowledge can be equated to the concept of Truth.
Anything that is known is, therefore, also true. It is also believed that
Knowledge, defined as we did, is supposed to be inapprehensible. In contrast,
Beliefs are potentially true and leads new hypothesis. Due to the structure of
deductive reasoning we can only start from hypothesis and test the deductive
conclusions on the reality. Whatever a scientific-security-theory is, it must
be defined starting from hypothesis.  A sound reasoning must lead to conclusion
that shall be tested on a real system. Thefore, only at the end of the reasoning
we will be able to test, and judge, the quality of the theory itself. The identification
of hypothesis should start from probably true beliefs (which means from
Information), and then from the state of the art. Instead of having a ``related work''
section, we review and cite the literature throughout the paper.


\section{Glossary -- A Formalization}\label{sec:glossary}
We now define a formalizations of the concepts described in
Section~\ref{sec:problem} and depicted in Figure~\ref{fig:safety-security}. We
base our formalization on first order logic (FOL) with the standard
truth-value semantics. The choice of this logic is required by the 
semantics of the concepts formalized afterwards in this section. 

We consider a first-order language $\follang$ over a
signature $\Sigma_P$ where $P,P',\ldots,P^n$ represent terms, and
$\varphi,\varphi',\ldots,\varphi^n$ represent formulas.
The syntax is defined as follows.
\begin{displaymath}
	\varphi := P~|~\neg\varphi~|~\varphi\wedge\varphi~|~\varphi\vee\varphi~|~\varphi\Rightarrow\varphi'~|~\forall P.~\varphi~|~\exists P.~\varphi
\end{displaymath}
where $\wedge$, $\neg$, $\vee$, and $\Rightarrow$ are connectives representing
conjunction, negation, disjunction, and (material) implication respectively; while
$\forall$ and $\exists$ represent the standard universal and existential
quantifiers (resp.). Finally, the symbol ``$.$'' is just syntactic sugar. 
We consider $\interpretation\subset\Phi\times\{\top,\bot\}$ as the interpretation
function, where $\Phi$ is any collection of sentences in $\follang$ 
and $\top$ and $\bot$ represent the concepts of ``Tautology/True''
and ``Contradiction/False'' respectively.

Following Figure~\ref{fig:safety-security}, we start by formalizing the outermost term: Causality.

\subsubsection{Causality Principle}\label{sec:causality}
We formalize the Causality Principle starting from a K modal
logic\autocite{Garson2018modal} (i.e. without restrictions on the causality
relation between worlds). The standard definition of K modal logic is given in
Definition~\ref{def:modallogic} in terms of an interpretation function (which
we named $\sigma$ with a slight abuse of notation) defined in
Definition~\ref{def:modalinterpretation}. 
\begin{definition}{\bf K Modal Logic --}\label{def:modallogic}
A K-frame is a frame $ \kframe=\langle \possibleworlds,\modalrelation \rangle $
	in the K modal logic where $ \modalrelation $ is the binary relation
	(i.e. a set of ordered pairs) between possible worlds
	$\modalrelation\subseteq\possibleworlds\times\possibleworlds$, where
	$\possibleworlds$ represents the possible worlds, and
	$\possibleworlds\neq\emptyset$. An actual world
	$\actualworld\in\possibleworlds$ is assumed. For any proposition $P$,
	an interpretation function $\interpretation(\world,P)$ returns the
	truth value of $P$; e.g. $\interpretation(\world,P)=\top$ means that
	$P$ holds in $\world$. A model is defined as the tuple
	$\kmodel=\langle\possibleworlds,\modalrelation,\interpretation\rangle$.
\end{definition}

\begin{definition}{\bf Causality as K Modal Interpretation Function --}\label{def:modalinterpretation}
	Causality is (recursively) defined as the modal interpretation function $\interpretation$, as follows. 
	\begin{enumerate}[noitemsep]
		\item[$(\interpretation0)$] if $\interpretation(\world,P)=\top$ then $\world\models P$
		\item[$(\interpretation1)$] $\world\models\neg P$ iff $\world\not\models P$
		\item[$(\interpretation2)$] $\world\models P \wedge Q$ iff $\world\models P$ and $\world\models Q$
		\item[$(\interpretation3)$] $\world\models\Box P$ iff for any world $\world'\in\possibleworlds$ if $\world \modalrelation\world'$ then $\world'\models P$
		\item[$(\interpretation4)$] $\world\models\Diamond P$ iff there exists a set of worlds $\World'\subset\possibleworlds$ such that for any $\world'\in\World'$, if $\world\modalrelation\world'$ then $\world'\models P$
		\item[$(\interpretation5)$] $\models P$ iff $\actualworld\models P$
	\end{enumerate}
	where truth is defined as necessary with $\Box$ and possible with $\Diamond$.
\end{definition}

%\begin{definition}{\bf Causality --}\label{def:causality}
%	The Causality principle is defined over a K Modal Logic (i.e. without
%	restrictions on the causality relation between worlds) as in
%	Definition~\ref{def:modallogic}, with the standard interpretation
%	function defined in~\ref{def:modalinterpretation}
%\end{definition}

The causality principle has been defined in its
generic form. In fact, the accessibility relation $\modalrelation$ is free from
any axiomatic restriction (e.g. it's not reflexive nor anti-reflexive).  
We will focus in Section~\ref{sec:process} (for our tests) on 
the application of our theory to CPS system engineering.
More detailed case studies (i.e. \fix{mr}{a CPS as a smart power-grid}) 
will be defined in Section~\ref{sec:prototest}, where
the accessibility relation defines in details how the system itself evolves, due to causal relation 
(i.e. by restricting the causality principle only to those cause-effect that defines
the system). Therefore, the definition of $\modalrelation$
will be specialized in a more strict way based on the application domains and case study.

We note that we have defined the
Causality principle without considering it a \emph{threat source} since 
we lack the concept of intent and maliciousness. Similarly, in the 
next sections we won't discriminate between Nature and Mankind until we introduce 
the concepts of maliciousness in Section~\ref{sec:theory} 
and then formally define a threat agent
in the same section.

\subsubsection{Agents: Mankind and Nature}\label{sec:mankind-nature}
\begin{table}[t]
\centering
\setlength{\tabcolsep}{3.5pt}
\renewcommand{\arraystretch}{1}
\scriptsize
\caption{RCC3, RCC5, and RCC8 relations between Regions $X$, $Y$ and $Z$ ~\label{tab:rcc358}~\label{tab:rcc}}
\begin{tabular}{ccclll} 
\rota{\textbf{RCC3}}&\rota{\textbf{RCC5}}&\rota{\textbf{RCC8}}&\textbf{Name} & \textbf{Notation} & \textbf{Definition} \\
\hline
&&&Connects with 			& $\mathit{C}(\mathit{X},\mathit{Y})$ 		& $\mathit{X}\subseteq \mathit{Y}$ \\
&&&Disconnected from		& $\neg \mathit{C}(\mathit{X},\mathit{Y})$		& $\mathit{X}\not\subseteq \mathit{Y}$\\
&&&Part of				& $\mathit{P}(\mathit{X},\mathit{Y})$		& $\forall \mathit{Z} ~\mathit{C}(\mathit{Z},\mathit{X}) \rightarrow \mathit{C}(\mathit{Z},\mathit{Y})$\\
&&&Overlaps			& $\mathit{O}(\mathit{X},\mathit{Y})$		& $\exists \mathit{Z} ~\mathit{P}(\mathit{Z},\mathit{X})\wedge \mathit{P}(\mathit{Z},\mathit{Y})$\\
\Tdot&&&  \textbf{Overlaps Not Equal} 	& $\mathit{ONE}(\mathit{X},\mathit{Y})$		& $\mathit{O}(\mathit{X},\mathit{Y}) \land \neg \mathit{EQ}(\mathit{X},\mathit{Y})$ \\
\Tdot&\Tdot&\Tdot& \textbf{Equal to} 		& $\mathit{EQ}(\mathit{X},\mathit{Y})$  		& $\mathit{P}(\mathit{X},\mathit{Y}) \wedge \mathit{P}(\mathit{Y},\mathit{X})$\\
\Tdot&\Tdot&\Tdot& \textbf{DiscRete from} 		& $\mathit{DR}(\mathit{X},\mathit{Y})$		& $\neg \mathit{O}(\mathit{X},\mathit{Y})$\\
&\Tdot&\Tdot&\textbf{Partial-Overlap}	& $\mathit{PO}(\mathit{X},\mathit{Y})$ 		& $\mathit{O}(\mathit{X},\mathit{Y})\wedge \neg \mathit{P}(\mathit{X},\mathit{Y}) \wedge \neg \mathit{P}(\mathit{Y},\mathit{X})$\\ 
&\Tdot&&\textbf{Proper-Part-of} 	& $\mathit{PP}(\mathit{X},\mathit{Y})$ 		& $\mathit{P}(\mathit{X},\mathit{Y})\wedge \neg \mathit{P}(\mathit{Y},\mathit{X})$\\ 
	&\Tdot&&\textbf{Proper-Part-of-\textit{\textbf{i}}nverse} & $\mathit{PPi}(\mathit{X},\mathit{Y})$ 		& $\mathit{P}(\mathit{Y},\mathit{X}) \wedge \neg \mathit{P}(\mathit{X},\mathit{Y})$\\
&&\Tdot&\textbf{Externally Connected} 	& $\mathit{EC}(\mathit{X},\mathit{Y})$ 		& $\mathit{C}(\mathit{X},\mathit{Y}) \wedge \neg\mathit{O}(\mathit{X},\mathit{Y})$\\ 
&&\Tdot&\textbf{Tangential PP} 	& $\mathit{TPP}(\mathit{X},\mathit{Y})$ 		& $\mathit{PP}(\mathit{X},\mathit{Y})\wedge\exists\mathit{Z}~[\mathit{EC}(\mathit{Z},\mathit{X}),\mathit{EC}(\mathit{Z},\mathit{Y})]$\\ 
&&\Tdot&\textbf{Tangential PPi} 	& $\mathit{TPPi}(\mathit{X},\mathit{Y})$ 		& $\mathit{TPP}(\mathit{Y},\mathit{X})$\\ 
&&\Tdot&\textbf{Non-Tangential PP} 	& $\mathit{NTPP}(\mathit{X},\mathit{Y})$ 		& $\mathit{PP}(\mathit{X},\mathit{Y})\wedge\neg\exists\mathit{Z}~[\mathit{EC}(\mathit{Z},\mathit{X}),\mathit{EC}(\mathit{Z},\mathit{Y})]$\\ 
&&\Tdot&\textbf{Non-Tangential PPi} 	& $\mathit{NTPPi}(\mathit{X},\mathit{Y})$ 		& $\mathit{NTPP}(\mathit{Y},\mathit{X})$\\ 
\end{tabular}
\end{table}

Mankind and Nature are defined in Section~\ref{sec:problem} as two abstract
agents, both as collections (i.e. an abstract type that does not imply a
specific implementation) of their sub-agents (i.e. humans for Mankind and
plants, animals, \&c.  for Nature). Similarly to\autocite{Santaca2016abf}, 
we define Mankind and Nature, and any
other agent in the reminder of this article, as a meronomy (an hierarchy of
Part-Whole relations) based on a standard definition of mereology, i.e. based
on the definition of Parthood relation between \emph{Parts}.  
However, shall 
consider different types of Part; so, we extend the mereology to a
mereo-topology\autocite{Smith1996mereotopology,Varzi1994mereotopology,Rachavelpula2017mereotopology},
to increase the number of different types considered and to generalize the relations between Parts (as
in Table~\ref{tab:rcc358}).
For the sake of readability, we use the term \emph{Region} both to refer to a
mereological Part and to a topological Region.
The choice of mereotopology is also correlated to the objective of defining a formal
ontology, which we use to define the (formal) semantics of the terms (Parts) in
Section~\ref{sec:problem}, and of the concepts of safety and security
(whole). We aim at creating a meronomy instead of the taxonomies such as the one provided
in\autocite{NIST2020NVD,MITRE2020CVE} or instead of the poorly justified 
CVSS\autocite{Mell2007CVSS} scoring system.  

A mereotopology, as defined e.g. in\autocite{Rachavelpula2017mereotopology},
is an ordered mathematical structure where the basic relation between Regions is the
reflexive and symmetric\fixnote{mr}{I guess it must be monotonic as defined in\autocite{Rachavelpula2017mereotopology} but I don't find it consistently in other papers.} 
\emph{Parthood} relation $\subseteq$. 
\begin{definition}{\bf Parthood --}\label{def:parthood}
Given any pair of mereotopological Regions $X$ and $Y$,
	\begin{enumerate}[noitemsep]
		\item Reflexivity: $\forall X.~ (X\subseteq X)$
		\item Symmetry: $\forall X, Y.~ (X\subseteq Y \Rightarrow Y\subseteq X)$
	\end{enumerate}
\end{definition}

As later defined in Definition~\ref{def:agent}, 
the Parthood relation orders a universe of agents $\agentuniverse$ by defining
the so called \emph{Connects with} (see in Table~\ref{tab:rcc358}) relation 
between Regions. We want this universe $\agentuniverse$ to be 
expressible in FOL. In this way, we
can reason both on the constituent of security (i.e. its terms and agents
defining a system where security needs to be considered), and on 
the evolution of those constituent w.r.t. cause-effects relations according
to the modal structure of causality we defined in Section~\ref{sec:causality}.
This will allow us (in Section~\ref{sec:process} and Section~\ref{sec:prototest})
a better\fixnote{mr}{Can we prove this by construction?} 
positioning w.r.t. risk assessment technologies
(which most often reason on the constituent of a system design), and
protocol verification tools (which requires some formalization of a cause-effect relation, e.g.
in Linear Temporal Logic).

As we argued in Section~\ref{sec:problem}, we must correlate the definition of
agent (i.e. Mankind and Nature) in Definition~\ref{def:mankind-nature} to the
mathematical structure of the logic that defines them.  We express Mankind and
Nature as formulas over the theory of mereology and then in terms of
mereotopological Regions, extending the interpretation function
$\interpretation$, to include a formal theory of mereotopology.  We use the
Region Connection Calculus (RCC), as defined
in~\cite{bennettLogics,improvingRCC}, to provide an axiomatization of the
spatial concepts and relations in first-order logic to correlate the algebraic
structure to mereology. In its broader definition, the RCC theory is composed
by eight axioms, and is known as RCC8. In the text, for brevity, 
we will often focus only on RCC5 (without
loss of generality) by not considering tangential connections between spatial
Regions. We discuss the choice of RCC5 in more detail in
Appendix~\ref{app:rcc}. In Table~\ref{tab:rcc}, we
summarize the axioms of the Region Connection Calculus (see, e.g., \autocite{Grutter2008rcc}).

\begin{definition}{\bf RCC axiomatization --}
	For any $X,Y$ pair of Regions in a mereotopology:
	\begin{itemize}[noitemsep]
	\item[$(\interpretation6)$] $\interpretation(X\subseteq Y)$ iff [$\interpretation(X\subseteq X)=\top$, and $\interpretation(X\subseteq Y)=\bot$ or $\interpretation(Y\subseteq X)=\top$]\fixnote{mr}{would it be better/clearer or just correct to write $\sigma(X\subseteq Y)$ iff $\sigma(X\subseteq X\wedge [X\subseteq Y \vee Y\subseteq X])=\top$}
	\item[$(\interpretation7)$] $\world\models\connects{X}{Y}$ iff $\world\models X\subseteq Y$ 
	\item[$(\interpretation8)$] $\world\models\disconnected{X}{Y}$ iff  $\world\not\models X\subseteq Y$
	\item[$(\interpretation9)$] $\world\models\partof{X}{Y}$ iff $\world\models\forall Z.~\connects{Z}{X}\Rightarrow\connects{Z}{Y}$
	\item[$(\interpretation10)$] $\world\models\overlaps{X}{Y}$ iff $\world\models\exists Z.~\partof{Z}{X}\wedge\partof{Z}{Y}$
	\item[$(\interpretation11)$] $\world\models\eq{X}{Y}$ iff $\world\models\partof{X}{Y}\wedge\partof{Y}{X}$
	\item[$(\interpretation12)$] $\world\models\dr{X}{Y}$ iff $\world\models\neg\overlaps{X}{Y}$
	\item[$(\interpretation13)$] $\world\models\po{X}{Y}$ iff $\world\models\overlaps{X}{Y}\wedge\neg\partof{X}{Y}\wedge\neg\partof{Y}{X}$
	\item[$(\interpretation14)$] $\world\models\pp{X}{Y}$ iff $\world\models\partof{X}{Y}\wedge\neg\partof{Y}{X}$
	\item[$(\interpretation15)$] $\world\models\ppi{X}{Y}$ iff $\world\models\partof{Y}{X}\wedge\neg\partof{X}{Y}$
\end{itemize}
where $Z$ is a mereotopological Region.
\end{definition}

\begin{definition}{\bf Agent: Mankind or Nature --}\label{def:agent}
	An agent $\agent\in\agentuniverse$ is a tuple
	$\langle\rcc(\region',\region''),\ldots,\rcc(\region^{n-1},\region^n)\rangle$
	of RCC relations $\rcc$ over mereotopological Regions
	${\region',\ldots,\region^n}\subseteq\Region$. 
\end{definition}

Currently, we do not distinguish between Mankind and Nature (since we still
lack of the definition of ``malicious intent'', which is defined in
Section~\ref{sec:theory}) and we have defined them as two generic
agents.

As depicted in Figure~\ref{fig:safety-security}, Causality, Mankind, and Nature
have a dashed border representing their correlation in terms of cause-effect
and then in terms of formal structure which defines them: Modal Logic.
Vulnerability, Defect, Incident, Accident, and Threat, similarly, 
are correlated (depicted as a solid border) in terms of underlying 
formal structure: Mereotopology.

\subsubsection{Regions: Vulnerability and Defect (and Weakness)}\label{sec:vulnerabilitydefect}
As defined in Section~\ref{sec:problem}, a Vulnerability or a Defect is a
\emph{Weakness}.  As an example, a categorization of weaknesses is given
in\autocite{MITRE2020CWEresearch} with 808 weaknesses categorized as ``Research
Concepts'', distributed as follows:
\begin{itemize}[noitemsep]
	\item Incorrect Calculation - ($682$)
	\item Incorrect Access of Indexable Resource (``Range Error'') - ($118$)
	\item Use of Insufficiently Random Values - ($330$)
	\item Improper Interaction Between Multiple Correctly-Behaving Entities - ($435$)
	\item Improper Control of a Resource Through its Lifetime - ($664$)
	\item Insufficient Control Flow Management - ($691$)
	\item Protection Mechanism Failure - ($693$)
	\item Incorrect Comparison - ($697$)
	\item Improper Check or Handling of Exceptional Conditions - ($703$)
	\item Improper Enforcement of Message or Data Structure - ($707$)
	\item Improper Adherence to Coding Standards - ($710$)
\end{itemize}

The definition given by the MITRE in\autocite{MITRE2020CWEweakness} of
weakness is: ``Software weaknesses are errors that can lead to software
vulnerabilities. A software vulnerability, such as those enumerated on the
Common Vulnerabilities and Exposures (CVE) List, is a mistake in software that
can be directly used by a hacker to gain access to a system or network''.
The definition is circular if we interpret the word ``error'' and
``mistake'' with the same semantics: a weakness is an error that leads to a
vulnerability and a vulnerability is a mistake which, in turn, is a weakness.
The only difference (between weakness and vulnerability) seems to be
that one can consider weakness as a ground term and state that a
vulnerability is caused by a weakness, i.e. 
$\World,\Weakness\models\Diamond\Vulnerability\wedge\Weakness$ where
$\Weakness,\Vulnerability$ are Regions of Weaknesses and Vulnerabilities (resp.);
accepting the hierarchy in the CWE\autocite{CWE} as ground truth. 
Similarly, we consider the CVE\autocite{CVE} (a database of Vulnerability) 
or the CVE reported in the NVD as a ground truth. 

\begin{definition}{\bf Region: Weakness and Vulnerability --}\label{def:weakness}
	A Region $\region\subseteq\agent$ of an agent
	$\agent\in\agentuniverse$, is defined as Weakness $\Weakness$ (i.e.
	representing a weakness introduced by the agent into any phase of
	production, e.g.  of the secure process development life-cycle, of any
	system or subsystem) iff there exists $\world\in\possibleworlds$ such
	that $\world,\Weakness\models\Diamond\exists\Weakness',\Vulnerability'.~
	\rcc(\Weakness',\Vulnerability')\wedge\neg\dr{\Weakness'}{\Vulnerability'}$, 
	where $\Weakness'\subseteq\Weakness$, and
	$\Vulnerability'\subseteq\Vulnerability$ is a Region of an agent $\agent'\in\agentuniverse$.
\end{definition}

\begin{example}{CWE-116: Improper Encoding or Escaping of Output\autocite{CWE-116} --}
	\begin{itemize} 
		\item Description: The software prepares a structured message
		for communication with another component, but encoding or
		escaping of the data is either missing or done incorrectly. As
		a result, the intended structure of the message is not
		preserved. 
		\item Example: This code displays an email
		address that was submitted as part of a form. Example language JSP.
		\begin{verbatim} 
		<% String email = request.getParameter("email"); %> 
		...
		Email Address: <%= email %>
		\end{verbatim}
		The value read from the form parameter is
		reflected back to the client browser without
		having been encoded prior to output, allowing
		various XSS attacks (CWE-79).
		\item Observed Examples
			\begin{itemize}
			\item CVE-2008-4636\autocite{CVE-2008-4636}: OS command injection in backup
				software using shell metacharacters in a
				filename; correct behavior would
				require that this filename could not be
				changed.
			\item CVE-2008-0769\autocite{CVE-2008-0769}: Web application does not set the
				charset when sending a page to a browser,
				allowing for XSS exploitation when a
				browser chooses an unexpected encoding.
			\item CVE-2008-0005\autocite{CVE-2008-0005}: Program does not set the charset
				when sending a page to a browser, allowing for
				XSS exploitation when a browser chooses
				an unexpected encoding.
			\item CVE-2008-5573\autocite{CVE-2008-5573}: SQL injection via password
				parameter; a strong password might contain \&
			\item CVE-2008-3773\autocite{CVE-2008-3773}: Cross-site scripting in chat
				application via a message subject, which
				normally might contain \& and other
				XSS-related characters.
			\item CVE-2008-0757\autocite{CVE-2008-0757}: Cross-site scripting in chat
				application via a message, which normally might
				be allowed to contain arbitrary
				content.
			\end{itemize}
\end{itemize}

\begin{figure}[t]
	\centering
	\begin{subfigure}[b]{.5\textwidth}
		\centering
		\includegraphics[width=.7\textwidth]{rcc5_CWE-116.pdf}
		\caption{Representation with RCC3}
		\label{fig:rcc5_CWE-116}
	\end{subfigure}\begin{subfigure}[b]{.5\textwidth}
		\centering
		\includegraphics[width=.7\textwidth]{rcc8_CWE-116.pdf}
		\caption{Representation with RCC8}
		\label{fig:rcc8_CWE-116}
	\end{subfigure}
	\begin{subfigure}[b]{\textwidth}
		\centering
		\includegraphics[width=\textwidth]{KML_CWE-116.pdf}
		\caption{Representation in K-Modal Logic}
		\label{fig:KML_CWE-116}
	\end{subfigure}
	\caption{Relations between CWE-116 and correlated CVEs}
\end{figure}
Definition~\ref{def:weakness} states that CWE-116 is a weakness iff there exists a world in
K Modal Logic, representing the system in which this weakness exists, 
such that the natural evolution of this system (i.e. formalized by the causality principle) 
make it possible to reach another state of the system (i.e. another, accessible world)
where there exist a vulnerability that can be implemented from CWE-116 (and this relation 
is not DR). The CWE website proposes the connection between CWE-116 and, for example, 
CVE-2008-5573; a vulnerability of the login sub-system 
of the Poll Pro v2.0\autocite{pollpro} system.
The formal relation between the two is given as a link (i.e. URI) between the CWE and
the CVE, the description of the relation is not defined but it is supposed to be
inferred from the descriptions of the CWE-CVE. 
We can formally represent this link as the ONE connection in RCC3, depicted in 
Figure~\ref{fig:rcc5_CWE-116}, or EC connection in RCC8 (Figure~\ref{fig:rcc8_CWE-116}).

\fixnote{mr}{shouldn't we consider the agent who introduces the weakness as dishonest? don't we say this in part 1?}
\end{example}

It is interesting to note that the CWE-CVE relation expresses the correlation
between weaknesses and vulnerabilities in the most simple form, such that we
can formalize all the relation as ONE in RCC3 or EC in RCC8.  To express a more
complex relation between the two we shall analyze the definition of CWE-116.
This is related to the Weakness-Vulnerability-Incident process (i.e. to the
details on the implementation/realization in Figure~\ref{fig:safety-security})
that we analyze in the next section, Section~\ref{sec:incidentaccident}.

\begin{definition}{\bf Region: Vulnerability and Defect --}\label{def:vulnerability-defect}
A region $\region\subseteq\agent$ of an agent $\agent\in\agentuniverse$ is
	called \emph{Vulnerability} if the agent $\agent$ is referred to as Mankind, 
	\emph{Defect} if the agent is referred to as Nature.
\end{definition}

\subsubsection{Process: Incidents and Accidents}\label{sec:incidentaccident}
In our informal definition depicted in Figure~\ref{fig:safety-security}, an
Incident is generated through a process caused by a Vulnerability (which,
in turn, is caused by a Weakness). Symmetrically, Nature has a process from Defect
to Accidents. We start by analyzing the process generated by Mankind, which is
divided into the following phases (as shown in Figure~\ref{fig:cwe-nvd-capec}):
\begin{enumerate}
	\item An agent (Mankind or Nature) defines a system, e.g. by following an engineering process for creating a new system, or for redesigning or improving an existing system (e.g. a component or a collection of components) 
	\item The dogmatic definition of a system contains a Weakness (i.e. the agent, Mankind or Nature, introduces an unintended behavior), for example as an un-intended behavior or as an architectural fallacy. The Weakness can be applied to the abstract definition of the system.
	\item The Weakness can be defined as a malicious process for a system, called Vulnerability (or Defect for Nature)
	\item The process expressed by the Vulnerability can be made operational for a specific implementation of the system (i.e. implemented by Mankind or realized by Nature) 
\end{enumerate}
Those additional details are depicted in Figure~\ref{fig:safety-security_2}, an update
of Figure~\ref{fig:safety-security}.

\begin{figure}[t]
	\centering
	\includegraphics[width=\textwidth]{safety-security_2.pdf}
	\caption{Updated overview security and safety keywords, where Process
	has to be intended as ``Abstract Exploitation Process''}
	\label{fig:safety-security_2}
\end{figure}

\begin{definition}{\bf Exploitation Process --} An \emph{Exploit} is an \emph{implementation} of
a Vulnerability, where a Vulnerability describes how to transform a
Weakness into an abstract malicious process, which can be applied\footnote{Meaning that the
malicious process, i.e.  any structured procedure such as a protocol, can be
somehow made operational, e.g. implemented} to one or many systems. So, the act
of an agent (Mankind) of ``exploiting a Vulnerability'' is the process of
making the Vulnerability operational, by implementing the Vulnerability w.r.t.
a specific target system.  Whenever the agent is Nature, the implementation is
more broadly considered a realization\footnote{``The state of being
realized''\autocite{Merriam2020realization}}. 
\end{definition}

\begin{figure}[t]
	\centering
	\includegraphics[width=\textwidth]{cwe-nvd-capec.pdf}
	\caption{Cybersecurity online dictionaries and connections}
	\label{fig:cwe-nvd-capec}
\end{figure}

The Common Attack Pattern Enumeration and
Classification\autocite{MITRE2020CAPEC} (CAPEC) online service provides a
``community resource'' which aims at ``identifying and understanding attacks''.
The MITRE writes on the CAPEC homepage that ``Understanding how the adversary
operates is essential to effective cyber security. CAPEC helps by providing a
comprehensive dictionary of known patterns of attack employed by adversaries to
exploit known weaknesses in cyber-enabled capabilities. It can be used by
analysts, developers, testers, and educators to advance community understanding
and enhance defenses.''. We can extrapolate two major concepts: 
\begin{itemize}
	\item While understanding the exploitation process is the aim of the
		CWE and CVE initiatives, understanding the hacking process is
		the aim of the CAPEC initiative.
	\item CAPEC provides a \emph{dictionary} of \emph{known} attack patterns.
		As stated in details in Section~\ref{sec:problem}, this dictionary contains
		empirical evidences and should not be used to induce a cybersecurity 
		theory; but it can be used to validate a theory.
\end{itemize}

\begin{example}
The CWE-116 doesn't just relate the Weakness to a number of Vulnerability in
	the CVE but also relates the Weakness to a number ($4$ for the CWE-116)
	of abstract attack path in the CAPEC:
	\begin{itemize}
		\item CAPEC-104\autocite{CAPEC-104} Cross Zone Scripting
		\item CAPEC-73\autocite{CAPEC-73} User-Controlled Filename
		\item CAPEC-81\autocite{CAPEC-81} Web Logs Tampering
		\item CAPEC-85\autocite{CAPEC-85} AJAX Fingerprinting
	\end{itemize}

	Each CAPEC entry has a reference to one or many CWE, including CWE-116.
\end{example}

After the exploitation process, as depicted in Figure~\ref{fig:cwe-nvd-capec},
another process starts and transforms a Vulnerability in an Incident by
Attacking the system with a malicious Hack, as follows.
\begin{enumerate}
\setcounter{enumi}{4}
	\item An agent (Mankind\footnote{For the sake of readability we focus
		only on Mankind as an agent}) the exploitation process can be
		used to hack a system. We use the term hack (as hacking) to
		stress that nothing prevents this process from being honest,
		however, for the sake of simplicity, we focus on Attacks which
		are malicious by definition.
	\item An Attack is applied to a system with malicious intent by an
		agent
	\item The application of the attack results in an Incident which
		impacts, i.e. poses a Threat to, an Asset
\end{enumerate}

\begin{definition}{\bf Black-hat Hacking Process}\label{def:hackingprocess}
An \emph{Attack} is the act of making operational an \emph{Exploitation Process} 
	causing an Incident. 
\end{definition}

\begin{example}	CAPEC-81\autocite{CAPEC-81} Web Logs Tampering --
	\begin{itemize}
		\item Description: An attacker is able to cause a victim to
			load content into their web-browser that bypasses
			security zone controls and gain access to increased
			privileges to execute scripting code or other web
			objects such as unsigned ActiveX controls or applets.
			This is a privilege elevation attack targeted at
			zone-based web-browser security. In a zone-based model,
			pages belong to one of a set of zones corresponding to
			the level of privilege assigned to that page. Pages in
			an untrusted zone would have a lesser level of access
			to the system and/or be restricted in the types of
			executable content it was allowed to invoke. In a
			cross-zone scripting attack, a page that should be
			assigned to a less privileged zone is granted the
			privileges of a more trusted zone. This can be
			accomplished by exploiting bugs in the browser,
			exploiting incorrect configuration in the zone
			controls, through a cross-site scripting attack that
			causes the attackers' content to be treated as coming
			from a more trusted page, or by leveraging some piece
			of system functionality that is accessible from both
			the trusted and less trusted zone. This attack differs
			from "Restful Privilege Escalation" in that the latter
			correlates to the inadequate securing of RESTful access
			methods (such as HTTP DELETE) on the server, while
			cross-zone scripting attacks the concept of security
			zones as implemented by a browser. 
		\item Execution Flow: 
			\begin{enumerate}
				\item Explore: Find systems susceptible to the
					attack: Find systems that contain
					functionality that is accessed from
					both the internet zone and the local
					zone. There needs to be a way to supply
					input to that functionality from the
					internet zone and that original input
					needs to be used later on a page from a
					local zone. 
				\item Experiment: Find the insertion point for
					the payload: The attacker first needs
					to find some system functionality or
					possibly another weakness in the system
					(e.g. susceptibility to cross site
					scripting) that would provide the
					attacker with a mechanism to deliver
					the payload (i.e. the code to be
					executed) to the user. The location
					from which this code is executed in the
					user's browser needs to be within the
					local machine zone.
				\item Exploit: Craft and inject the payload:
					Develop the payload to be executed in
					the higher privileged zone in the
					user's browser. Inject the payload and
					attempt to lure the victim (if
					possible) into executing the
					functionality which unleashes the
					payload.
			\end{enumerate}
		\item Prerequisite: The target must be using a zone-aware browser. 
		\item Consequences: 
			\begin{itemize}
				\item Integrity: Modify Data
				\item Confidentiality: Read Data
				\item Confidentiality, Access Control, Authorization: Gain Privileges
				\item Confidentiality, Integrity, Availability: Execute Unauthorized Commands
			\end{itemize}
	\end{itemize}

\end{example}

The formalization of the Exploitation and Black-hat Hacking processes is the
formalization of two foundational cybersecurity concepts which lay the basis
for the definition (and formalization) of the cybersecurity theory in the next
Section (Section~\ref{sec:theory}). The formal definition of those
processes shall:
\begin{itemize}
	\item in Section~\ref{sec:theory}, formally define a system as a structure of agents, and
	\item in Section~\ref{sec:process}, correlate the formal definition of agents (i.e. the
		Definition~\ref{def:agent}) with the Causality principle (i.e.
		the formalization of the Kripke structure in
		Definition~\ref{def:modallogic}). 
\end{itemize}


\printbibliography

\end{document}
