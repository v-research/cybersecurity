In this paper, we proposed a foundational theory on security, arguing that
security-related issues are not related to the maliciousness of an agent but to
the vagueness of the security controls on the engineering processes.
The current state-of-the-art processes allows, given a specification,
an engineer to design the system in such a way that security issues arise due to the lack
of proper security risk assessment processes.
Those design, again, lack of security verification processes based on a solid 
security foundational theory and then permit the generation of insecure implementation.
The security verification and test-case generation will be the focus of our next steps.

We may conclude that the problem of security is a problem related to the 
many possible design and, in turn, implementation given a specification.
Philosophically, the problem is similar to the epistemological search for truth,
where the challenge is to relate Information and Belief to Human Knowledge.
Scientifically, generalizing the Human to an Agent, the problem is
to relate Assertions and Behaviors, to Facts. From an engineering standpoint,
by generalizing the concept of Agents as architectural subsystems, the
problem is to link Channels (and Ports) and Functional Architectures to 
Requirements.
