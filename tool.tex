\begin{itemize}
\item structured in package where the \emph{architecture} is
\begin{itemize}
\item is defined by an object diagram where the \emph{physical architecture} is structured on dependency connections and
\item the \emph{functional architecture} is linked to the agents and defined as nested/inner object diagrams
\item the information flow between functional or physical components is explicit 
\end{itemize}
\item port to socket must exist and have the same name
\item socket to port may not (case loopback)
\item a base defines the granularity. if a base is a bit, it faithfully? represents information. This is important 
	for the definitino of Vulnerability. If a port has all the weaknesses but only 1 base is transferred over that port,
		then ony the DR can be exploited into a vulnerability since a PO would be equal to EQ
\end{itemize}

\subsection{Impact and Likelihood}\label{sec:impact}
As described in~\autocite{CORASMethod}, in order to quantitatively estimate the
cybersecurity \emph{risk} of a CPS design, one needs to estimate both the
\emph{likelihood} of \emph{attack paths} (attack from now on)\fixnote{mr}{check terminology
consistency} along with their (negative) \emph{impact} on a requirement which, in
turn, is related to some \emph{company assets}. An attack that has a negative impact
on a requirement invalidates the requirement, leading to one or more \emph{incidents}.
